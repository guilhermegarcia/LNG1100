% \makeatletter \def\NR@nopatch@beamer{} \makeatother
\documentclass[xcolor=dvipsnames, onlymath, 10pt, aspectratio=169, handout]{beamer}
\usecolortheme[named=Black]{structure}
% \usepackage{etex} % Too many packages

\usepackage[utf8]{inputenc}
\usepackage[T1]{fontenc}
\usepackage{tipa}
\usepackage{color}
\usepackage{natbib}

\usepackage[french,english]{babel}

\usepackage{multicol}
\usepackage{caption}
\usepackage{vowel}
\usepackage{diagbox}
\usepackage{fontawesome}
\usepackage{color}
\usepackage{tikz-qtree}
\usepackage{booktabs}
\usepackage{hyperref}
\usepackage{bibentry}
\usepackage{arydshln}
\usepackage{linguex}
\usetikzlibrary{arrows.meta, shapes, arrows, positioning, calc, trees}
\usepackage{pifont}
\usepackage{textgreek}
% \usepackage{eulervm}
\usepackage{libertinus}
\usepackage{inconsolata}
\usepackage{datetime}

\definecolor{pastelorange}{RGB}{255,179,71}


% ============ >>>> MAC

\input{/Users/gdgarcia/Dropbox/Academia/LaTeX/garcia_latex_commands.tex}


% COURSE:

\newcommand{\course}{LNG-1100 : Méthodes expérimentales\\et analyse de données}

% TOPIC:

\newcommand{\classtopic}{\vspace{2ex}Exploration de données}

% DATE:

\newcommand{\classdate}{\fbox{3}}

% =====================================
% =====================================
% =====================================


% Define header colors:
\definecolor{lav1}{RGB}{184, 0, 35}
\definecolor{lav2}{RGB}{246, 195, 67}




\setbeamertemplate{itemize item}{$\bullet$}
\setbeamertemplate{itemize subitem}{$\circ$}
\setbeamerfont{frametitle}{series=\bfseries}
\setbeamercolor{frametitle}{fg=lav,bg=white}
\setbeamerfont{title}{series=\bfseries, parent=structure}
\setbeamercolor{title}{fg=lav,bg=white}



\usepackage[most]{tcolorbox}

    \tcbset{
        enhanced,
        colback=lavb!5!white,
        boxrule=0.1pt,
        colframe=lavb!80!white,
        fonttitle=\bfseries,
        width=5cm, box align=top,
        nobeforeafter
       }

\renewcommand\bibsection{\section[]{\refname}}
\usetheme{boxes}            % Simple and clear
\setbeamercolor{button}{bg=black,fg=white}
\beamertemplatenavigationsymbolsempty  % Remove nav controls




\title{\course}

\subtitle{\classtopic}

\author{Guilherme D.\ Garcia}


\institute[Université Laval] % (optional, but mostly needed)
{
  \mylink{https://fr.gdgarcia.ca}{fr.gdgarcia.ca}\vspace{5ex}
}


\date{\classdate}





\subject{Linguistique}


\begin{document}


\begin{frame}
\vspace{2ex}
\textcolor{lav1}{\noindent\rule{0.66\textwidth}{3pt}}
\textcolor{lav2}{\noindent\rule{0.33\textwidth}{3pt}}

  \titlepage

\vfill

\begin{center}
{\includegraphics[width=2.5cm]{/Users/gdgarcia/Dropbox/Academia/Admin/Uni-logos/ULaval1.png}}   	
\end{center}

\end{frame}

\addtobeamertemplate{navigation symbols}{}{%
    \usebeamerfont{footline}%
    \usebeamercolor[fg]{footline}%
    \hspace{1em}%
    \vspace{1ex}%
    {\insertframenumber} sur {\inserttotalframenumber}
}



\setbeamertemplate{background}{\tikz[overlay,remember picture]\node[opacity=0.4, xshift=-1cm, yshift=.8cm] at (current page.south east){\includegraphics[width=.75cm]{/Users/gdgarcia/Dropbox/Academia/Admin/Uni-logos/ULaval2bw.png}};}

% ==================================
%  
%
% 
%    
%\end{frame}

% =======================

% =======================


% ==================

\begin{frame}{Questionnaire}{\mylink{https://forms.office.com/r/X6zdX4pws1}{forms.office.com/r/X6zdX4pws1}}

\begin{center}
	\includegraphics[width = 0.4\textwidth]{qr.png}
\end{center}
	
% Some questions related to the reading material (1--7 Barnier 2023); intro to R etc.
	
\end{frame}


% =======================

\begin{frame}{Plan de la séance}{Dans RStudio aujourd'hui}
	
	\begin{enumerate}
		\item La notion de \emph{tidy data}
		\item Intro à l'exploration de données avec \var{dplyr} et \var{ggplot2} \citep[chapitres 8, 9, 10, 12]{barnier_R}
		\item Pratique
	\end{enumerate}
	
\end{frame}

% ==================



% =================

\appendix
\begin{frame}[allowframebreaks]

{
\footnotesize
        \frametitle{Références}
        \bibliographystyle{apalike}
          \bibliography{../../../../../References/references}

}

\end{frame}



%%%%%%%%%%%%%%%%%%%%%%%%%% APPENDIX

%%%%%%%%%%%%%%%%%%%%%%%%%%%%%%%%%%%%%%%%%%%%%%%%


%\begin{frame}{Annexe i}
%
%\begin{center}
%	...
%\end{center}
%  
%\end{frame}


\end{document}
